\RequirePackage[ngerman=ngerman-x-latest]{hyphsubst}
\documentclass[english,ngerman,BCOR=6mm,cdgeometry=no,DIV=13,twoside,cdmath=false]{tudscrreprt}
\usepackage{selinput}\SelectInputMappings{adieresis={ä},germandbls={ß}}
\usepackage{scrhack} % To fix some of the warnings regarding floats

\usepackage[T1]{fontenc}

\usepackage[main=english]{babel}
%\usepackage[utf8x]{inputenc}
%\usepackage{isodate}

\usepackage[table]{xcolor}
\definecolor{LinkColor}{rgb}{0,0,0.3}


% typical mathesymbols
\usepackage{amsmath}
\usepackage{amsthm}
\usepackage{amssymb}
\usepackage{textcomp}
\usepackage{latexsym}
\usepackage{mathtools}
\usepackage{wasysym}



\usepackage[cmbright]{sfmath}


% to allow subfigures inside the figure Environment
\usepackage{subfig}
% insert images
\usepackage{graphicx}
% make all include graphics correspond to width and aspectratio
\setkeys{Gin}{width=\linewidth,keepaspectratio}

% use bibliography
\usepackage[authoryear]{natbib}

%allow urls in the code
% Definitionen von Farben

\usepackage{hyperref}
\hypersetup{colorlinks=true,%
           linkcolor=LinkColor,%
           citecolor=LinkColor,%
           filecolor=LinkColor,%
           menucolor=LinkColor,%
           %pagecolor=LinkColor,%
           urlcolor=LinkColor}

% insert sourcecode
\usepackage{listings}
% package for advanzed enumerations
\usepackage{enumerate}

%Packages and declarations for acronyms
\usepackage[single,sort]{acro}
\DeclareAcronym{RML}{
short = RML, long = Role Modeling Language%
, short-plural = s%
, long-plural = s%
}



%Preamble generated from md
\usepackage{pbox} \usepackage{xspace} \usepackage{semantic}
\newcommand{\I}{\ensuremath{\mathcal{I}}\xspace}
\newcommand{\R}{\ensuremath{\mathcal{R}}\xspace}
\newcommand{\true}{\ensuremath{\mathsf{T}}\xspace}
\newcommand{\false}{\ensuremath{\mathsf{F}}\xspace}
\newcommand{\nat}{\ensuremath{\mathbb{N}}\xspace}
\DeclareMathOperator{\rimpl}{-\!{\triangleright}}
\DeclareMathOperator{\requi}{\triangleleft-\!{\triangleright}}
\DeclareMathOperator{\rproh}{\shortparallel}
\DeclareMathOperator{\rdont}{\sim} \newtheorem{definition}{Definition}
\newtheorem{example}{Example} \usepackage{blindtext}


%allow todos in the text
\newcommand{\todo}[1]{\fbox{\scriptsize\textbf{#1}}}
\newcommand{\todoln}[1]{\fbox{\scriptsize\parbox{.9\linewidth}{\textbf{#1}}}}

\begin{document}
\faculty{Faculty of Computer Science}
\department{}
\institute{Institute of Software and Multimedia Technology}
\chair{Software Technology Group}
\date{30.9.2000}
\author{Nathan Fillion}
\title{A Family of Role Modeling Languages}
\thesis{diss}
\graduation[Dr.-Ing.]{Doktor-Ingenieur}
\dateofbirth{01.01.1970}
\placeofbirth{Moscow}



\referee{Robert Picardo}

\professor{Patrick Stewart\and Leonard Nimoy}



%Use \thanks{footnotetext} for footnotes on the title page

\dedication{For my loving children}

%Turn of TUD style for part pages
\renewcommand*{\partpagestyle}{plain}



\maketitle

%add acknowledgements

\confirmation



\TUDoption{abstract}{section,multiple}
\begin{abstract}[pagestyle=plain]
\blindtext
\nextabstract[ngerman]
\blindtext
\end{abstract}

\tableofcontents

\part{State of the Art}

\chapter{Introduction}\label{introduction}

\ac{RML} is awesome \citep{kuehn2014}. \acfp{RML} are even better
\citep{kuehn2014}.

\section{Motivation}\label{motivation}

\blindtext

\section{Background}\label{background}

\blindtext

\section{Problem Definition}\label{problem-definition}

\blindtext

\section{Outline}\label{outline}

\chapter{Prerequisites}\label{prerequisites}

\section{First Order Logic}\label{first-order-logic}

\blindtext

\section{Feature Modeling}\label{feature-modeling}

\blindtext

\section{Language Product Lines}\label{language-product-lines}

\chapter{Background}\label{background-1}

\blindtext

\section{The Nature of Roles}\label{the-nature-of-roles}

\blindtext

\subsection{Behavioral Nature}\label{behavioral-nature}

\blindtext

\subsection{Relational Nature}\label{relational-nature}

\blindtext

\subsection{Context-Dependent Nature}\label{context-dependent-nature}

\part{Solution}

\chapter{Formal Role-based Modeling
Language}\label{formal-role-based-modeling-language}

\section{Formalization}\label{formalization}

\subsection{Type Level}\label{type-level}

After introducing the ontological foundations and the graphical
notation, we can introduce our formal model, starting on the type level.
For brevity, we omitted the notion of \emph{attributes} from these
definitions. Nevertheless, the necessary additions are presented in the
Appendix.

\begin{definition}[Compartment Role Object Model]
Let $NT$, $RT$, $CT$, and $RST$ be mutual disjoint sets of 
Natural Types, Role Types, Compartment Types, and Relationship Types, respectively.
Then a \emph{Compartment Role Object Model (CROM)} is a tuple $\mathcal{M}=(NT, RT, CT,RST,\text{fills}, \text{parts}, \text{rel})$ where
$\text{fills} \subseteq (NT \cup CT) \times RT$ is a relation,
$\text{parts} : CT \rightarrow 2^{RT} $ and
$\text{rel} : RST \rightarrow (RT \times RT$) are total functions.
A CROM is denoted \emph{well-formed} if the following axioms hold:
\begin{align}
\forall rt \in RT   &\ \exists t \in (NT \cup CT) : (t,rt) \in \text{fills}\\
\forall ct \in CT   &: \text{parts}(ct) \neq \emptyset\\
\forall rt \in RT   &\ \exists ! ct \in CT : rt \in \text{parts}(ct)\\
\forall rst \in RST &: \text{rel}(rst) = (rt_1,rt_2) \wedge rt_1 \neq rt_2\\
\forall rst \in RST &\ \exists ct \in CT : \text{rel}(rst) = (rt_1,rt_2)\ \wedge\nonumber\\
                    &\ rt_1,rt_2 \in \text{parts}(ct)
\end{align} 
\end{definition}

\noindent
In detail, \emph{fills} denotes that rigid types can play roles of a
certain role type, \emph{parts} maps compartment types to their
contained role types, and \emph{rel} captures the two role types at the
respective ends of each relationship type. The well-formedness rules
ensure that the \emph{fills}-relation is surjective (1); each
compartment type has a nonempty, disjoint set of role types as its parts
(2, 3); and \emph{rel} maps each relationship type to exactly two
distinct role types of the same compartment type (4, 5).

\begin{example}[Compartment Role Object Model]
Let $\mathcal{B} = (NT, RT, CT, RST,\text{fills}, \text{parts}, \text{rel})$ be the model of the bank,
where the idividual components are defined as follows:
\begin{align*}
NT \coloneqq \{ &\text{Person}, \text{Company}, \text{Account} \}\\
RT \coloneqq \{ &\text{Customer}, \text{Consultant},\text{CA},\text{SA},\text{Source},\text{Target},\\
                &\text{MoneyTransfer} \}\\
CT = \{ &\text{Bank}, \text{Transaction} \}\\
RST = \{ &\text{own\_ca}, \text{own\_sa}, \text{advises}, \text{trans} \}\\
\mathit{fills} \coloneqq \{ &(\text{Person}, \text{Consultant}), (\text{Person}, \text{Customer}),%\\
\end{align*}
%$NT \coloneqq \{ \text{Person}, \text{Company}, \text{Account} \}$ is the set of natural types,\linebreak
%$RT \coloneqq \{ \text{Customer}, \text{Consultant},\text{CA},\text{SA},\text{Source},\text{Target},$ $\text{MoneyTransfer} \}$ the set of role types,
%$CT = \{\text{Bank},$ $\text{Transaction} \}$ the set of compartment types, and
%$RST = \{\text{own\_ca},$ $\text{own\_sa},$ $\text{advises},$ $\text{trans} \}$ the set of relationship types.
%The other elements are specified as follows:
\begin{align*}
            & (\text{Company}, \text{Customer}), (\text{Account}, \text{Source}),\\
            & (\text{Account}, \text{Target}), (\text{Account}, \text{CA}),\\
            & (\text{Account}, \text{SA}), (\text{Transaction}, \text{MoneyTransfer}) \}\\
 \mathit{parts} \coloneqq \{ & \text{Bank} \rightarrow \{\text{Consultant},\text{Customer},\text{CA},\text{SA},\\
                                                        &\ \text{MoneyTransfer} \},\\
                                          & \text{Transaction} \rightarrow \{ \text{Source},\text{Target}\} \}\\
 \mathit{rel} \coloneqq \{ & \text{own\_ca} \rightarrow (\text{Customer},\text{CA}),\\
                           & \text{own\_sa} \rightarrow (\text{Customer},\text{SA}),\\
                           & \text{advises} \rightarrow (\text{Consultant},\text{Customer}),\\
                           & \text{trans} \rightarrow (\text{Source},\text{Target}) \}
\end{align*}
\end{example}

\noindent
The bank model $\mathcal{B}$ is simply created in four steps. First, all
the natural types, compartment types, role types, and relationship types
are collected into the corresponding set.\footnotemark[1]
 Second, the set of role types contained in each compartment type is
assigned to the \emph{parts}-function. Third, it is specified which
natural type can \emph{fill} which role type, and finally the
\emph{rel}-function is defined for the role types at the ends of each
relationship type. Thus, CROMs can be retrieved from their graphical
representation. The presented bank model $\mathcal{B}$ is well-formed,
because each defined role type is filled by at least one natural type or
compartment type (1), each compartment type consists of a non-empty (2)
and disjoint (3) set of role types, and each relationship type is
established between two distinct role types (4) of the same compartment
type (5).

\subsection{Instance Level}\label{instance-level}

\subsection{Constraint Level}\label{constraint-level}

\chapter{First Family of Role-based Modeling
Editors}\label{first-family-of-role-based-modeling-editors}

\section{Architecture}\label{architecture}

\section{Configuration}\label{configuration}

\section{Family of Model
Transformations}\label{family-of-model-transformations}

\chapter{Conclusion}\label{conclusion}

\section{Summery}\label{summery}

\section{Related Work}\label{related-work}

\section{Contributions}\label{contributions}

\section{Future Research}\label{future-research}

\subsection{Adding Time to the Formal
Model}\label{adding-time-to-the-formal-model}

\listoffigures

\listoftables

\lstlistoflistings

% add list of abbriviation
\printacronyms[sort,heading=chapter*,name=List of Abbreviations]

\nocite{*}
\renewcommand{\bibname}{References}
\bibliographystyle{apalike}
\bibliography{thesis.bib}



\end{document}
